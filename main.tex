%% SEG-2021 Poster Theme
%% ---------------------
%% This is the renewed IMAGE (SEG) 2021 Poster template. It is
%% modified from Radboud University corporate style. 
%% See the original work here:
%% https://www.overleaf.com/latex/templates/the-beamer-poster-version-of-the-radboud-university-corporate-style-poster-template/mfddtqfwpspm

\documentclass[roundedcorners=true, titleposition=left]{beamerSEG2021poster}
%% The class takes the following optional arguments:
%% roundedcorners: true, false (default=false)
%% titleposition: left, center, right (default=right)

\institute[ECE]{Department of Electrical and Computer Engineering, University of Houston, TX, United States}
\title{The Beamer poster template of SEG-2021, University of Houston}
\date{\today}
\author{lama-fan}

%%%%%%%%%%%%%%%%%%%%%%%%%%%%%%%%%%%%%%%%%%%%%%%%%%%%%%%%%%%%%%%%%%%%%%%%%%%%%%%%%%%%%%
\begin{document}
\begin{frame}
\begin{columns}
\begin{column}{0.19\textwidth}
\begin{beamercolorbox}[center, wd=\textwidth]{postercolumn}
\begin{minipage}[T]{0.95\textwidth}
\parbox[t][\columnheight]{\textwidth}{%
  \begin{block}{My first block!}
  	Hoi
  \end{block}
  \medskip
  \begin{block}{Disclaimer}
  	This is a very early version, but some of us already need it quite soon. Anyway, let me know if there are any problems at \texttt{l.onrust@let.ru.nl}.
  \end{block}
 	\medskip
    \begin{block}{How to use this stuff to create a poster}
    	Stop using this if you are not comfortable with \LaTeX. In the other case, proceed with caution.
        
        This file (\texttt{main.tex}) contains the presentation. If consists of two \texttt{column}s in a \texttt{columns} environment. Each column then consists of multiple blocks, separated by whatever you think is suitable (\texttt{medskip}, \texttt{bigskip}, \texttt{vfill}, \ldots). 
    \end{block}
}
\end{minipage}
\end{beamercolorbox}
\end{column}

\begin{column}{0.19\textwidth}
\begin{beamercolorbox}[center, wd=\textwidth]{postercolumn}
\begin{minipage}[T]{0.95\textwidth}
\parbox[t][\columnheight]{\textwidth}{%
  \begin{block}{Formulae}
    	We approximate the integral with samples $\{\mathcal{P}^{(i)}, \boldsymbol\Theta^{(i)}\}_{i=1}^I$ drawn from $p(\mathcal{P}, \boldsymbol\Theta|\mathcal{D})$: 
  	\begin{equation}
    	p(w|\mathbf{u},\mathcal{D})\approx\sum_{i=1}^{I}p(w|\mathbf{u},\mathcal{P}^{(i)},\boldsymbol\Theta^{(i)})
        \end{equation}
  and $p(w|\mathbf{u}, \mathcal{P},\boldsymbol\Theta)$ is given by the recursive function with  $p(w|\pi(\emptyset),\mathcal{P},\boldsymbol\Theta) = 1/V$ and 
  \begin{equation}
  	\begin{split}
    	p(w|\mathbf{u},\mathcal{P},\boldsymbol\Theta) = &\frac{c_{\mathbf{u}w\cdot}-d_{|\mathbf{u}|}t_{\mathbf{u}w\cdot}}{\theta_{|\mathbf{u}|}+c_{\mathbf{u}\cdot\cdot}} \\
        &+ \frac{\theta_{|\mathbf{u}|}+d_{|\mathbf{u}|}t_{\mathbf{u}\cdot\cdot}}{\theta_{|\mathbf{u}|}+c_{\mathbf{u}\cdot\cdot}} p(w|\pi(\mathbf{u}),\mathcal{P},\boldsymbol\Theta),
        \end{split}
        \end{equation}
        where the counts in partition $P_{\mathbf{u}}$ correspond to $G_{\mathbf{u}}$.
    \end{block}
    \vfill
    \begin{block}{Drake}
    	Started from the bottom.
    \end{block}
}
\end{minipage}
\end{beamercolorbox}
\end{column}

\begin{column}{0.19\textwidth}
\begin{beamercolorbox}[center, wd=\textwidth]{postercolumn}
\begin{minipage}[T]{0.95\textwidth}
\parbox[t][\columnheight]{\textwidth}{%
  \begin{block}{My second block!}
  	Hoi! We refer a paper from here \cite{lamport,kopka}.
  \end{block}
  \medskip
  \begin{block}{My funniest block!}
    \begin{figure}[H]
      \centering
      \includegraphics[width=0.95\textwidth]{funnygirl}
      \DeclareGraphicsExtensions.
      \caption{An example of figure.}
    \end{figure}
  \end{block}
  \medskip
  \begin{block}{And the obligatory boring block\ldots}
    \begin{table}[H]
      \centering
      \caption{An example of table.}
    	\begin{tabular}{lllll}
        \toprule
             & jrc  & 1bw   & emea & wp   \\
        \midrule
        jrc  & 3.65 & 10.22 & 9.91 & 9.98 \\
        1bws & 9.58 & 7.31  & 9.89 & 8.94 \\
        emea & 9.23 & 10.16 & 1.88 & 9.72 \\
        wps  & 9.12 & 8.83  & 9.97 & 7.76 \\
        \bottomrule
      \end{tabular}
    \end{table}
  \end{block}
}
\end{minipage}
\end{beamercolorbox}
\end{column}

\begin{column}{0.19\textwidth}
\begin{beamercolorbox}[center, wd=\textwidth]{postercolumn}
\begin{minipage}[T]{0.95\textwidth}
\parbox[t][\columnheight]{\textwidth}{%
  \begin{block}{The previous title made no sense}
  	\begin{itemize}
    	\item a
        \item b
        \item c
        \item d
        \item e
        \item f
        \item g
        \item h
        \item i
        \item j
        \item k
    \end{itemize}
  \end{block}
}
\end{minipage}
\end{beamercolorbox}
\end{column}

\begin{column}{0.19\textwidth}
\begin{beamercolorbox}[center, wd=\textwidth]{postercolumn}
\begin{minipage}[T]{0.95\textwidth}
\parbox[t][\columnheight]{\textwidth}{%
  \begin{block}{The previous title made no sense}
  	\begin{itemize}
    	\item a
        \item b
        \item c
        \item d
        \item e
        \item f
        \item g
        \item h
        \item i
        \item j
        \item k
    \end{itemize}
  \end{block}
  \medskip
  \begin{block}{Reference}
    \begin{columns}[T]
      \begin{column}{\textwidth}
        \vspace{-.2ex}
        \footnotesize
        \bibliographystyle{IEEEtran}
        % argument is your BibTeX string definitions and bibliography database(s)
        \bibliography{bib/example}
      \end{column}
    \end{columns}
  \end{block}
}
\end{minipage}
\end{beamercolorbox}
\end{column}

\end{columns}
\end{frame}

\end{document}

